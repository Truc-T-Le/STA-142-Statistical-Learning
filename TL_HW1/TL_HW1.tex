\documentclass[11pt]{article}

    \usepackage[breakable]{tcolorbox}
    \usepackage{parskip} % Stop auto-indenting (to mimic markdown behaviour)
    
    \usepackage{iftex}
    \ifPDFTeX
    	\usepackage[T1]{fontenc}
    	\usepackage{mathpazo}
    \else
    	\usepackage{fontspec}
    \fi

    % Basic figure setup, for now with no caption control since it's done
    % automatically by Pandoc (which extracts ![](path) syntax from Markdown).
    \usepackage{graphicx}
    % Maintain compatibility with old templates. Remove in nbconvert 6.0
    \let\Oldincludegraphics\includegraphics
    % Ensure that by default, figures have no caption (until we provide a
    % proper Figure object with a Caption API and a way to capture that
    % in the conversion process - todo).
    \usepackage{caption}
    \DeclareCaptionFormat{nocaption}{}
    \captionsetup{format=nocaption,aboveskip=0pt,belowskip=0pt}

    \usepackage[Export]{adjustbox} % Used to constrain images to a maximum size
    \adjustboxset{max size={0.9\linewidth}{0.9\paperheight}}
    \usepackage{float}
    \floatplacement{figure}{H} % forces figures to be placed at the correct location
    \usepackage{xcolor} % Allow colors to be defined
    \usepackage{enumerate} % Needed for markdown enumerations to work
    \usepackage{geometry} % Used to adjust the document margins
    \usepackage{amsmath} % Equations
    \usepackage{amssymb} % Equations
    \usepackage{textcomp} % defines textquotesingle
    % Hack from http://tex.stackexchange.com/a/47451/13684:
    \AtBeginDocument{%
        \def\PYZsq{\textquotesingle}% Upright quotes in Pygmentized code
    }
    \usepackage{upquote} % Upright quotes for verbatim code
    \usepackage{eurosym} % defines \euro
    \usepackage[mathletters]{ucs} % Extended unicode (utf-8) support
    \usepackage{fancyvrb} % verbatim replacement that allows latex
    \usepackage{grffile} % extends the file name processing of package graphics 
                         % to support a larger range
    \makeatletter % fix for grffile with XeLaTeX
    \def\Gread@@xetex#1{%
      \IfFileExists{"\Gin@base".bb}%
      {\Gread@eps{\Gin@base.bb}}%
      {\Gread@@xetex@aux#1}%
    }
    \makeatother

    % The hyperref package gives us a pdf with properly built
    % internal navigation ('pdf bookmarks' for the table of contents,
    % internal cross-reference links, web links for URLs, etc.)
    \usepackage{hyperref}
    % The default LaTeX title has an obnoxious amount of whitespace. By default,
    % titling removes some of it. It also provides customization options.
    \usepackage{titling}
    \usepackage{longtable} % longtable support required by pandoc >1.10
    \usepackage{booktabs}  % table support for pandoc > 1.12.2
    \usepackage[inline]{enumitem} % IRkernel/repr support (it uses the enumerate* environment)
    \usepackage[normalem]{ulem} % ulem is needed to support strikethroughs (\sout)
                                % normalem makes italics be italics, not underlines
    \usepackage{mathrsfs}
    

    
    % Colors for the hyperref package
    \definecolor{urlcolor}{rgb}{0,.145,.698}
    \definecolor{linkcolor}{rgb}{.71,0.21,0.01}
    \definecolor{citecolor}{rgb}{.12,.54,.11}

    % ANSI colors
    \definecolor{ansi-black}{HTML}{3E424D}
    \definecolor{ansi-black-intense}{HTML}{282C36}
    \definecolor{ansi-red}{HTML}{E75C58}
    \definecolor{ansi-red-intense}{HTML}{B22B31}
    \definecolor{ansi-green}{HTML}{00A250}
    \definecolor{ansi-green-intense}{HTML}{007427}
    \definecolor{ansi-yellow}{HTML}{DDB62B}
    \definecolor{ansi-yellow-intense}{HTML}{B27D12}
    \definecolor{ansi-blue}{HTML}{208FFB}
    \definecolor{ansi-blue-intense}{HTML}{0065CA}
    \definecolor{ansi-magenta}{HTML}{D160C4}
    \definecolor{ansi-magenta-intense}{HTML}{A03196}
    \definecolor{ansi-cyan}{HTML}{60C6C8}
    \definecolor{ansi-cyan-intense}{HTML}{258F8F}
    \definecolor{ansi-white}{HTML}{C5C1B4}
    \definecolor{ansi-white-intense}{HTML}{A1A6B2}
    \definecolor{ansi-default-inverse-fg}{HTML}{FFFFFF}
    \definecolor{ansi-default-inverse-bg}{HTML}{000000}

    % commands and environments needed by pandoc snippets
    % extracted from the output of `pandoc -s`
    \providecommand{\tightlist}{%
      \setlength{\itemsep}{0pt}\setlength{\parskip}{0pt}}
    \DefineVerbatimEnvironment{Highlighting}{Verbatim}{commandchars=\\\{\}}
    % Add ',fontsize=\small' for more characters per line
    \newenvironment{Shaded}{}{}
    \newcommand{\KeywordTok}[1]{\textcolor[rgb]{0.00,0.44,0.13}{\textbf{{#1}}}}
    \newcommand{\DataTypeTok}[1]{\textcolor[rgb]{0.56,0.13,0.00}{{#1}}}
    \newcommand{\DecValTok}[1]{\textcolor[rgb]{0.25,0.63,0.44}{{#1}}}
    \newcommand{\BaseNTok}[1]{\textcolor[rgb]{0.25,0.63,0.44}{{#1}}}
    \newcommand{\FloatTok}[1]{\textcolor[rgb]{0.25,0.63,0.44}{{#1}}}
    \newcommand{\CharTok}[1]{\textcolor[rgb]{0.25,0.44,0.63}{{#1}}}
    \newcommand{\StringTok}[1]{\textcolor[rgb]{0.25,0.44,0.63}{{#1}}}
    \newcommand{\CommentTok}[1]{\textcolor[rgb]{0.38,0.63,0.69}{\textit{{#1}}}}
    \newcommand{\OtherTok}[1]{\textcolor[rgb]{0.00,0.44,0.13}{{#1}}}
    \newcommand{\AlertTok}[1]{\textcolor[rgb]{1.00,0.00,0.00}{\textbf{{#1}}}}
    \newcommand{\FunctionTok}[1]{\textcolor[rgb]{0.02,0.16,0.49}{{#1}}}
    \newcommand{\RegionMarkerTok}[1]{{#1}}
    \newcommand{\ErrorTok}[1]{\textcolor[rgb]{1.00,0.00,0.00}{\textbf{{#1}}}}
    \newcommand{\NormalTok}[1]{{#1}}
    
    % Additional commands for more recent versions of Pandoc
    \newcommand{\ConstantTok}[1]{\textcolor[rgb]{0.53,0.00,0.00}{{#1}}}
    \newcommand{\SpecialCharTok}[1]{\textcolor[rgb]{0.25,0.44,0.63}{{#1}}}
    \newcommand{\VerbatimStringTok}[1]{\textcolor[rgb]{0.25,0.44,0.63}{{#1}}}
    \newcommand{\SpecialStringTok}[1]{\textcolor[rgb]{0.73,0.40,0.53}{{#1}}}
    \newcommand{\ImportTok}[1]{{#1}}
    \newcommand{\DocumentationTok}[1]{\textcolor[rgb]{0.73,0.13,0.13}{\textit{{#1}}}}
    \newcommand{\AnnotationTok}[1]{\textcolor[rgb]{0.38,0.63,0.69}{\textbf{\textit{{#1}}}}}
    \newcommand{\CommentVarTok}[1]{\textcolor[rgb]{0.38,0.63,0.69}{\textbf{\textit{{#1}}}}}
    \newcommand{\VariableTok}[1]{\textcolor[rgb]{0.10,0.09,0.49}{{#1}}}
    \newcommand{\ControlFlowTok}[1]{\textcolor[rgb]{0.00,0.44,0.13}{\textbf{{#1}}}}
    \newcommand{\OperatorTok}[1]{\textcolor[rgb]{0.40,0.40,0.40}{{#1}}}
    \newcommand{\BuiltInTok}[1]{{#1}}
    \newcommand{\ExtensionTok}[1]{{#1}}
    \newcommand{\PreprocessorTok}[1]{\textcolor[rgb]{0.74,0.48,0.00}{{#1}}}
    \newcommand{\AttributeTok}[1]{\textcolor[rgb]{0.49,0.56,0.16}{{#1}}}
    \newcommand{\InformationTok}[1]{\textcolor[rgb]{0.38,0.63,0.69}{\textbf{\textit{{#1}}}}}
    \newcommand{\WarningTok}[1]{\textcolor[rgb]{0.38,0.63,0.69}{\textbf{\textit{{#1}}}}}
    
    
    % Define a nice break command that doesn't care if a line doesn't already
    % exist.
    \def\br{\hspace*{\fill} \\* }
    % Math Jax compatibility definitions
    \def\gt{>}
    \def\lt{<}
    \let\Oldtex\TeX
    \let\Oldlatex\LaTeX
    \renewcommand{\TeX}{\textrm{\Oldtex}}
    \renewcommand{\LaTeX}{\textrm{\Oldlatex}}
    % Document parameters
    % Document title
    \title{TL\_HW1}
    
    
    
    
    
% Pygments definitions
\makeatletter
\def\PY@reset{\let\PY@it=\relax \let\PY@bf=\relax%
    \let\PY@ul=\relax \let\PY@tc=\relax%
    \let\PY@bc=\relax \let\PY@ff=\relax}
\def\PY@tok#1{\csname PY@tok@#1\endcsname}
\def\PY@toks#1+{\ifx\relax#1\empty\else%
    \PY@tok{#1}\expandafter\PY@toks\fi}
\def\PY@do#1{\PY@bc{\PY@tc{\PY@ul{%
    \PY@it{\PY@bf{\PY@ff{#1}}}}}}}
\def\PY#1#2{\PY@reset\PY@toks#1+\relax+\PY@do{#2}}

\@namedef{PY@tok@w}{\def\PY@tc##1{\textcolor[rgb]{0.73,0.73,0.73}{##1}}}
\@namedef{PY@tok@c}{\let\PY@it=\textit\def\PY@tc##1{\textcolor[rgb]{0.25,0.50,0.50}{##1}}}
\@namedef{PY@tok@cp}{\def\PY@tc##1{\textcolor[rgb]{0.74,0.48,0.00}{##1}}}
\@namedef{PY@tok@k}{\let\PY@bf=\textbf\def\PY@tc##1{\textcolor[rgb]{0.00,0.50,0.00}{##1}}}
\@namedef{PY@tok@kp}{\def\PY@tc##1{\textcolor[rgb]{0.00,0.50,0.00}{##1}}}
\@namedef{PY@tok@kt}{\def\PY@tc##1{\textcolor[rgb]{0.69,0.00,0.25}{##1}}}
\@namedef{PY@tok@o}{\def\PY@tc##1{\textcolor[rgb]{0.40,0.40,0.40}{##1}}}
\@namedef{PY@tok@ow}{\let\PY@bf=\textbf\def\PY@tc##1{\textcolor[rgb]{0.67,0.13,1.00}{##1}}}
\@namedef{PY@tok@nb}{\def\PY@tc##1{\textcolor[rgb]{0.00,0.50,0.00}{##1}}}
\@namedef{PY@tok@nf}{\def\PY@tc##1{\textcolor[rgb]{0.00,0.00,1.00}{##1}}}
\@namedef{PY@tok@nc}{\let\PY@bf=\textbf\def\PY@tc##1{\textcolor[rgb]{0.00,0.00,1.00}{##1}}}
\@namedef{PY@tok@nn}{\let\PY@bf=\textbf\def\PY@tc##1{\textcolor[rgb]{0.00,0.00,1.00}{##1}}}
\@namedef{PY@tok@ne}{\let\PY@bf=\textbf\def\PY@tc##1{\textcolor[rgb]{0.82,0.25,0.23}{##1}}}
\@namedef{PY@tok@nv}{\def\PY@tc##1{\textcolor[rgb]{0.10,0.09,0.49}{##1}}}
\@namedef{PY@tok@no}{\def\PY@tc##1{\textcolor[rgb]{0.53,0.00,0.00}{##1}}}
\@namedef{PY@tok@nl}{\def\PY@tc##1{\textcolor[rgb]{0.63,0.63,0.00}{##1}}}
\@namedef{PY@tok@ni}{\let\PY@bf=\textbf\def\PY@tc##1{\textcolor[rgb]{0.60,0.60,0.60}{##1}}}
\@namedef{PY@tok@na}{\def\PY@tc##1{\textcolor[rgb]{0.49,0.56,0.16}{##1}}}
\@namedef{PY@tok@nt}{\let\PY@bf=\textbf\def\PY@tc##1{\textcolor[rgb]{0.00,0.50,0.00}{##1}}}
\@namedef{PY@tok@nd}{\def\PY@tc##1{\textcolor[rgb]{0.67,0.13,1.00}{##1}}}
\@namedef{PY@tok@s}{\def\PY@tc##1{\textcolor[rgb]{0.73,0.13,0.13}{##1}}}
\@namedef{PY@tok@sd}{\let\PY@it=\textit\def\PY@tc##1{\textcolor[rgb]{0.73,0.13,0.13}{##1}}}
\@namedef{PY@tok@si}{\let\PY@bf=\textbf\def\PY@tc##1{\textcolor[rgb]{0.73,0.40,0.53}{##1}}}
\@namedef{PY@tok@se}{\let\PY@bf=\textbf\def\PY@tc##1{\textcolor[rgb]{0.73,0.40,0.13}{##1}}}
\@namedef{PY@tok@sr}{\def\PY@tc##1{\textcolor[rgb]{0.73,0.40,0.53}{##1}}}
\@namedef{PY@tok@ss}{\def\PY@tc##1{\textcolor[rgb]{0.10,0.09,0.49}{##1}}}
\@namedef{PY@tok@sx}{\def\PY@tc##1{\textcolor[rgb]{0.00,0.50,0.00}{##1}}}
\@namedef{PY@tok@m}{\def\PY@tc##1{\textcolor[rgb]{0.40,0.40,0.40}{##1}}}
\@namedef{PY@tok@gh}{\let\PY@bf=\textbf\def\PY@tc##1{\textcolor[rgb]{0.00,0.00,0.50}{##1}}}
\@namedef{PY@tok@gu}{\let\PY@bf=\textbf\def\PY@tc##1{\textcolor[rgb]{0.50,0.00,0.50}{##1}}}
\@namedef{PY@tok@gd}{\def\PY@tc##1{\textcolor[rgb]{0.63,0.00,0.00}{##1}}}
\@namedef{PY@tok@gi}{\def\PY@tc##1{\textcolor[rgb]{0.00,0.63,0.00}{##1}}}
\@namedef{PY@tok@gr}{\def\PY@tc##1{\textcolor[rgb]{1.00,0.00,0.00}{##1}}}
\@namedef{PY@tok@ge}{\let\PY@it=\textit}
\@namedef{PY@tok@gs}{\let\PY@bf=\textbf}
\@namedef{PY@tok@gp}{\let\PY@bf=\textbf\def\PY@tc##1{\textcolor[rgb]{0.00,0.00,0.50}{##1}}}
\@namedef{PY@tok@go}{\def\PY@tc##1{\textcolor[rgb]{0.53,0.53,0.53}{##1}}}
\@namedef{PY@tok@gt}{\def\PY@tc##1{\textcolor[rgb]{0.00,0.27,0.87}{##1}}}
\@namedef{PY@tok@err}{\def\PY@bc##1{{\setlength{\fboxsep}{\string -\fboxrule}\fcolorbox[rgb]{1.00,0.00,0.00}{1,1,1}{\strut ##1}}}}
\@namedef{PY@tok@kc}{\let\PY@bf=\textbf\def\PY@tc##1{\textcolor[rgb]{0.00,0.50,0.00}{##1}}}
\@namedef{PY@tok@kd}{\let\PY@bf=\textbf\def\PY@tc##1{\textcolor[rgb]{0.00,0.50,0.00}{##1}}}
\@namedef{PY@tok@kn}{\let\PY@bf=\textbf\def\PY@tc##1{\textcolor[rgb]{0.00,0.50,0.00}{##1}}}
\@namedef{PY@tok@kr}{\let\PY@bf=\textbf\def\PY@tc##1{\textcolor[rgb]{0.00,0.50,0.00}{##1}}}
\@namedef{PY@tok@bp}{\def\PY@tc##1{\textcolor[rgb]{0.00,0.50,0.00}{##1}}}
\@namedef{PY@tok@fm}{\def\PY@tc##1{\textcolor[rgb]{0.00,0.00,1.00}{##1}}}
\@namedef{PY@tok@vc}{\def\PY@tc##1{\textcolor[rgb]{0.10,0.09,0.49}{##1}}}
\@namedef{PY@tok@vg}{\def\PY@tc##1{\textcolor[rgb]{0.10,0.09,0.49}{##1}}}
\@namedef{PY@tok@vi}{\def\PY@tc##1{\textcolor[rgb]{0.10,0.09,0.49}{##1}}}
\@namedef{PY@tok@vm}{\def\PY@tc##1{\textcolor[rgb]{0.10,0.09,0.49}{##1}}}
\@namedef{PY@tok@sa}{\def\PY@tc##1{\textcolor[rgb]{0.73,0.13,0.13}{##1}}}
\@namedef{PY@tok@sb}{\def\PY@tc##1{\textcolor[rgb]{0.73,0.13,0.13}{##1}}}
\@namedef{PY@tok@sc}{\def\PY@tc##1{\textcolor[rgb]{0.73,0.13,0.13}{##1}}}
\@namedef{PY@tok@dl}{\def\PY@tc##1{\textcolor[rgb]{0.73,0.13,0.13}{##1}}}
\@namedef{PY@tok@s2}{\def\PY@tc##1{\textcolor[rgb]{0.73,0.13,0.13}{##1}}}
\@namedef{PY@tok@sh}{\def\PY@tc##1{\textcolor[rgb]{0.73,0.13,0.13}{##1}}}
\@namedef{PY@tok@s1}{\def\PY@tc##1{\textcolor[rgb]{0.73,0.13,0.13}{##1}}}
\@namedef{PY@tok@mb}{\def\PY@tc##1{\textcolor[rgb]{0.40,0.40,0.40}{##1}}}
\@namedef{PY@tok@mf}{\def\PY@tc##1{\textcolor[rgb]{0.40,0.40,0.40}{##1}}}
\@namedef{PY@tok@mh}{\def\PY@tc##1{\textcolor[rgb]{0.40,0.40,0.40}{##1}}}
\@namedef{PY@tok@mi}{\def\PY@tc##1{\textcolor[rgb]{0.40,0.40,0.40}{##1}}}
\@namedef{PY@tok@il}{\def\PY@tc##1{\textcolor[rgb]{0.40,0.40,0.40}{##1}}}
\@namedef{PY@tok@mo}{\def\PY@tc##1{\textcolor[rgb]{0.40,0.40,0.40}{##1}}}
\@namedef{PY@tok@ch}{\let\PY@it=\textit\def\PY@tc##1{\textcolor[rgb]{0.25,0.50,0.50}{##1}}}
\@namedef{PY@tok@cm}{\let\PY@it=\textit\def\PY@tc##1{\textcolor[rgb]{0.25,0.50,0.50}{##1}}}
\@namedef{PY@tok@cpf}{\let\PY@it=\textit\def\PY@tc##1{\textcolor[rgb]{0.25,0.50,0.50}{##1}}}
\@namedef{PY@tok@c1}{\let\PY@it=\textit\def\PY@tc##1{\textcolor[rgb]{0.25,0.50,0.50}{##1}}}
\@namedef{PY@tok@cs}{\let\PY@it=\textit\def\PY@tc##1{\textcolor[rgb]{0.25,0.50,0.50}{##1}}}

\def\PYZbs{\char`\\}
\def\PYZus{\char`\_}
\def\PYZob{\char`\{}
\def\PYZcb{\char`\}}
\def\PYZca{\char`\^}
\def\PYZam{\char`\&}
\def\PYZlt{\char`\<}
\def\PYZgt{\char`\>}
\def\PYZsh{\char`\#}
\def\PYZpc{\char`\%}
\def\PYZdl{\char`\$}
\def\PYZhy{\char`\-}
\def\PYZsq{\char`\'}
\def\PYZdq{\char`\"}
\def\PYZti{\char`\~}
% for compatibility with earlier versions
\def\PYZat{@}
\def\PYZlb{[}
\def\PYZrb{]}
\makeatother


    % For linebreaks inside Verbatim environment from package fancyvrb. 
    \makeatletter
        \newbox\Wrappedcontinuationbox 
        \newbox\Wrappedvisiblespacebox 
        \newcommand*\Wrappedvisiblespace {\textcolor{red}{\textvisiblespace}} 
        \newcommand*\Wrappedcontinuationsymbol {\textcolor{red}{\llap{\tiny$\m@th\hookrightarrow$}}} 
        \newcommand*\Wrappedcontinuationindent {3ex } 
        \newcommand*\Wrappedafterbreak {\kern\Wrappedcontinuationindent\copy\Wrappedcontinuationbox} 
        % Take advantage of the already applied Pygments mark-up to insert 
        % potential linebreaks for TeX processing. 
        %        {, <, #, %, $, ' and ": go to next line. 
        %        _, }, ^, &, >, - and ~: stay at end of broken line. 
        % Use of \textquotesingle for straight quote. 
        \newcommand*\Wrappedbreaksatspecials {% 
            \def\PYGZus{\discretionary{\char`\_}{\Wrappedafterbreak}{\char`\_}}% 
            \def\PYGZob{\discretionary{}{\Wrappedafterbreak\char`\{}{\char`\{}}% 
            \def\PYGZcb{\discretionary{\char`\}}{\Wrappedafterbreak}{\char`\}}}% 
            \def\PYGZca{\discretionary{\char`\^}{\Wrappedafterbreak}{\char`\^}}% 
            \def\PYGZam{\discretionary{\char`\&}{\Wrappedafterbreak}{\char`\&}}% 
            \def\PYGZlt{\discretionary{}{\Wrappedafterbreak\char`\<}{\char`\<}}% 
            \def\PYGZgt{\discretionary{\char`\>}{\Wrappedafterbreak}{\char`\>}}% 
            \def\PYGZsh{\discretionary{}{\Wrappedafterbreak\char`\#}{\char`\#}}% 
            \def\PYGZpc{\discretionary{}{\Wrappedafterbreak\char`\%}{\char`\%}}% 
            \def\PYGZdl{\discretionary{}{\Wrappedafterbreak\char`\$}{\char`\$}}% 
            \def\PYGZhy{\discretionary{\char`\-}{\Wrappedafterbreak}{\char`\-}}% 
            \def\PYGZsq{\discretionary{}{\Wrappedafterbreak\textquotesingle}{\textquotesingle}}% 
            \def\PYGZdq{\discretionary{}{\Wrappedafterbreak\char`\"}{\char`\"}}% 
            \def\PYGZti{\discretionary{\char`\~}{\Wrappedafterbreak}{\char`\~}}% 
        } 
        % Some characters . , ; ? ! / are not pygmentized. 
        % This macro makes them "active" and they will insert potential linebreaks 
        \newcommand*\Wrappedbreaksatpunct {% 
            \lccode`\~`\.\lowercase{\def~}{\discretionary{\hbox{\char`\.}}{\Wrappedafterbreak}{\hbox{\char`\.}}}% 
            \lccode`\~`\,\lowercase{\def~}{\discretionary{\hbox{\char`\,}}{\Wrappedafterbreak}{\hbox{\char`\,}}}% 
            \lccode`\~`\;\lowercase{\def~}{\discretionary{\hbox{\char`\;}}{\Wrappedafterbreak}{\hbox{\char`\;}}}% 
            \lccode`\~`\:\lowercase{\def~}{\discretionary{\hbox{\char`\:}}{\Wrappedafterbreak}{\hbox{\char`\:}}}% 
            \lccode`\~`\?\lowercase{\def~}{\discretionary{\hbox{\char`\?}}{\Wrappedafterbreak}{\hbox{\char`\?}}}% 
            \lccode`\~`\!\lowercase{\def~}{\discretionary{\hbox{\char`\!}}{\Wrappedafterbreak}{\hbox{\char`\!}}}% 
            \lccode`\~`\/\lowercase{\def~}{\discretionary{\hbox{\char`\/}}{\Wrappedafterbreak}{\hbox{\char`\/}}}% 
            \catcode`\.\active
            \catcode`\,\active 
            \catcode`\;\active
            \catcode`\:\active
            \catcode`\?\active
            \catcode`\!\active
            \catcode`\/\active 
            \lccode`\~`\~ 	
        }
    \makeatother

    \let\OriginalVerbatim=\Verbatim
    \makeatletter
    \renewcommand{\Verbatim}[1][1]{%
        %\parskip\z@skip
        \sbox\Wrappedcontinuationbox {\Wrappedcontinuationsymbol}%
        \sbox\Wrappedvisiblespacebox {\FV@SetupFont\Wrappedvisiblespace}%
        \def\FancyVerbFormatLine ##1{\hsize\linewidth
            \vtop{\raggedright\hyphenpenalty\z@\exhyphenpenalty\z@
                \doublehyphendemerits\z@\finalhyphendemerits\z@
                \strut ##1\strut}%
        }%
        % If the linebreak is at a space, the latter will be displayed as visible
        % space at end of first line, and a continuation symbol starts next line.
        % Stretch/shrink are however usually zero for typewriter font.
        \def\FV@Space {%
            \nobreak\hskip\z@ plus\fontdimen3\font minus\fontdimen4\font
            \discretionary{\copy\Wrappedvisiblespacebox}{\Wrappedafterbreak}
            {\kern\fontdimen2\font}%
        }%
        
        % Allow breaks at special characters using \PYG... macros.
        \Wrappedbreaksatspecials
        % Breaks at punctuation characters . , ; ? ! and / need catcode=\active 	
        \OriginalVerbatim[#1,codes*=\Wrappedbreaksatpunct]%
    }
    \makeatother

    % Exact colors from NB
    \definecolor{incolor}{HTML}{303F9F}
    \definecolor{outcolor}{HTML}{D84315}
    \definecolor{cellborder}{HTML}{CFCFCF}
    \definecolor{cellbackground}{HTML}{F7F7F7}
    
    % prompt
    \makeatletter
    \newcommand{\boxspacing}{\kern\kvtcb@left@rule\kern\kvtcb@boxsep}
    \makeatother
    \newcommand{\prompt}[4]{
        \ttfamily\llap{{\color{#2}[#3]:\hspace{3pt}#4}}\vspace{-\baselineskip}
    }
    

    
    % Prevent overflowing lines due to hard-to-break entities
    \sloppy 
    % Setup hyperref package
    \hypersetup{
      breaklinks=true,  % so long urls are correctly broken across lines
      colorlinks=true,
      urlcolor=urlcolor,
      linkcolor=linkcolor,
      citecolor=citecolor,
      }
    % Slightly bigger margins than the latex defaults
    
    \geometry{verbose,tmargin=1in,bmargin=1in,lmargin=1in,rmargin=1in}
    
    

\begin{document}
    
    \maketitle
    
    

    
    \hypertarget{sta-141c-hw-1}{%
\section{STA 141C HW \# 1}\label{sta-141c-hw-1}}

\hypertarget{truc-le}{%
\subsection{Truc Le}\label{truc-le}}

    \begin{tcolorbox}[breakable, size=fbox, boxrule=1pt, pad at break*=1mm,colback=cellbackground, colframe=cellborder]
\prompt{In}{incolor}{2}{\boxspacing}
\begin{Verbatim}[commandchars=\\\{\}]
\PY{k+kn}{import} \PY{n+nn}{pandas} \PY{k}{as} \PY{n+nn}{pd}
\PY{k+kn}{import} \PY{n+nn}{numpy} \PY{k}{as} \PY{n+nn}{np}
\PY{k+kn}{import} \PY{n+nn}{matplotlib}\PY{n+nn}{.}\PY{n+nn}{pyplot} \PY{k}{as} \PY{n+nn}{plt}
\end{Verbatim}
\end{tcolorbox}

    \hypertarget{problem-1}{%
\section{Problem 1}\label{problem-1}}

    \hypertarget{a}{%
\subsection{A)}\label{a}}

    Since we know the true relationship between X and Y are linear, we would
expect the least square line to be a close approximation of the true
linear regression line. Since the RSS measures the variance of the
regression model, we would expect the training RSS for the linear
regression to be lower than the RSS for the cubic regression.

    \hypertarget{b}{%
\subsection{B)}\label{b}}

    We do not have enough information to tell which RSS would give a lower
value in this case, since we do not know what the testing data is.
However, if we use the same logic as (A), we would assume the testing
RSS for the linear regression to be lower than the RSS for the cubic
regression due to the true relationship between X and Y being linear.

    \hypertarget{c}{%
\subsection{C)}\label{c}}

    In this case, we know that the true relationship between X and Y are not
linear, we would want to use a regression model that give us the most
flexibility when fitting a least square line. Therefore, we would expect
the cubic regression model to have a better fit to the data points than
the linear regression model, hence, the training cubic regression would
have a lower RSS in this case.

    \hypertarget{d}{%
\section{D)}\label{d}}

    Since we do not know what the testing data look like, therefore, we do
not have enough information to say which regressions would be lower.
However, if we use the same logic as (C) and make the assumption that
the true relationship between X and Y is not linear, we would then
expect the testing RSS cubic regression would be lower due to its high
flexibility in fitting the data.

    \hypertarget{problem-2}{%
\section{Problem 2}\label{problem-2}}

    \hypertarget{a}{%
\subsection{A)}\label{a}}

    I first generated 100 random uniform xs and 100 random normally
distributed variance. I then calculated the y data points using the
linear regression model by plugging in the generated xs and epsilon
random variables I previously calculated. From there, I plotted the
scatter plot of my x and y variables and fitted a least square line onto
the graph.

    \begin{tcolorbox}[breakable, size=fbox, boxrule=1pt, pad at break*=1mm,colback=cellbackground, colframe=cellborder]
\prompt{In}{incolor}{7}{\boxspacing}
\begin{Verbatim}[commandchars=\\\{\}]
\PY{n}{np}\PY{o}{.}\PY{n}{random}\PY{o}{.}\PY{n}{seed}\PY{p}{(}\PY{l+m+mi}{0}\PY{p}{)}
\PY{c+c1}{\PYZsh{} generating 100 random samples from the uniform distribution  }
\PY{n}{xs} \PY{o}{=} \PY{n}{np}\PY{o}{.}\PY{n}{random}\PY{o}{.}\PY{n}{uniform}\PY{p}{(}\PY{l+m+mi}{0}\PY{p}{,}\PY{l+m+mi}{1}\PY{p}{,}\PY{l+m+mi}{100}\PY{p}{)}


\PY{c+c1}{\PYZsh{} generating 100 random samples from the normal distribution  }
\PY{n}{es} \PY{o}{=} \PY{n}{np}\PY{o}{.}\PY{n}{random}\PY{o}{.}\PY{n}{normal}\PY{p}{(}\PY{l+m+mi}{0}\PY{p}{,} \PY{l+m+mi}{1}\PY{p}{,} \PY{l+m+mi}{100}\PY{p}{)}
\PY{n}{ys} \PY{o}{=} \PY{l+m+mi}{5} \PY{o}{+} \PY{l+m+mi}{3}\PY{o}{*}\PY{n}{xs} \PY{o}{+} \PY{n}{np}\PY{o}{.}\PY{n}{random}\PY{o}{.}\PY{n}{normal}\PY{p}{(}\PY{l+m+mi}{0}\PY{p}{,} \PY{l+m+mi}{1}\PY{p}{,} \PY{l+m+mi}{100}\PY{p}{)}

\PY{p}{(}\PY{n}{b1}\PY{p}{,} \PY{n}{b}\PY{p}{)}\PY{p}{,} \PY{p}{(}\PY{n}{SSE}\PY{p}{,}\PY{p}{)}\PY{p}{,} \PY{o}{*}\PY{n}{\PYZus{}} \PY{o}{=} \PY{n}{np}\PY{o}{.}\PY{n}{polyfit}\PY{p}{(}\PY{n}{xs}\PY{p}{,} \PY{n}{ys}\PY{p}{,} \PY{n}{deg}\PY{o}{=}\PY{l+m+mi}{1}\PY{p}{,} \PY{n}{full}\PY{o}{=}\PY{k+kc}{True}\PY{p}{)}

\PY{c+c1}{\PYZsh{} Scatterplot with the generated random samples with least square line}
\PY{n}{plt}\PY{o}{.}\PY{n}{plot}\PY{p}{(}\PY{n}{xs}\PY{p}{,} \PY{n}{b1}\PY{o}{*}\PY{n}{xs}\PY{o}{+}\PY{n}{b}\PY{p}{)}
\PY{n}{plt}\PY{o}{.}\PY{n}{scatter}\PY{p}{(}\PY{n}{xs}\PY{p}{,} \PY{n}{ys}\PY{p}{,}\PY{p}{)}
\PY{n}{plt}\PY{o}{.}\PY{n}{show}\PY{p}{(}\PY{p}{)}
\end{Verbatim}
\end{tcolorbox}

    \begin{center}
    \adjustimage{max size={0.9\linewidth}{0.9\paperheight}}{output_14_0.png}
    \end{center}
    { \hspace*{\fill} \\}
    
    \hypertarget{b}{%
\subsection{B)}\label{b}}

    I created a function that would calculate the \(\hat\beta_{1}\)
coefficients 1000 times and stored all of them into a list and return
the list. I then calculated the average of the \(\hat\beta_{1}\) by
summing all of the values in the list and dividing it by 1000. From
there, I plotted a histogram using the 1000 \(\hat\beta_{1}\)
coefficient values and found that the mean of the histogram is around
the same value as the calculated average value.

    The calculated value of the means of the \(\hat\beta_{1}\) coefficients
is 3.0010314639108975

    The histogram graph for the 1000 \(\hat\beta_{1}\) with calculating
epsilon using the normal distribution method:

    \begin{tcolorbox}[breakable, size=fbox, boxrule=1pt, pad at break*=1mm,colback=cellbackground, colframe=cellborder]
\prompt{In}{incolor}{3}{\boxspacing}
\begin{Verbatim}[commandchars=\\\{\}]
\PY{c+c1}{\PYZsh{}Generating 1000 sample using normal distribution to calculate epsilon and uniform distribution to calculate xi}
\PY{k}{def} \PY{n+nf}{linear\PYZus{}reg\PYZus{}norm}\PY{p}{(}\PY{n}{mu}\PY{p}{,} \PY{n}{sigma}\PY{p}{,} \PY{n}{size}\PY{p}{)}\PY{p}{:}
    \PY{n}{b} \PY{o}{=} \PY{p}{[}\PY{p}{]}
    \PY{n}{np}\PY{o}{.}\PY{n}{random}\PY{o}{.}\PY{n}{seed}\PY{p}{(}\PY{l+m+mi}{1}\PY{p}{)}
    \PY{k}{for} \PY{n}{i} \PY{o+ow}{in} \PY{n+nb}{range}\PY{p}{(}\PY{l+m+mi}{1000}\PY{p}{)}\PY{p}{:}
        \PY{n}{x} \PY{o}{=} \PY{n}{np}\PY{o}{.}\PY{n}{random}\PY{o}{.}\PY{n}{uniform}\PY{p}{(}\PY{n}{mu}\PY{p}{,}\PY{n}{sigma}\PY{p}{,}\PY{n}{size}\PY{p}{)}
        \PY{n}{e} \PY{o}{=} \PY{n}{np}\PY{o}{.}\PY{n}{random}\PY{o}{.}\PY{n}{normal}\PY{p}{(}\PY{n}{mu}\PY{p}{,} \PY{n}{sigma}\PY{p}{,} \PY{n}{size}\PY{p}{)}
        \PY{n}{y} \PY{o}{=} \PY{l+m+mi}{5} \PY{o}{+} \PY{l+m+mi}{3}\PY{o}{*}\PY{n}{x} \PY{o}{+} \PY{n}{e}
        \PY{p}{(}\PY{n}{b1\PYZus{}hat}\PY{p}{,} \PY{n}{b\PYZus{}hat}\PY{p}{)}\PY{p}{,} \PY{p}{(}\PY{n}{SSE}\PY{p}{,}\PY{p}{)}\PY{p}{,} \PY{o}{*}\PY{n}{\PYZus{}} \PY{o}{=} \PY{n}{np}\PY{o}{.}\PY{n}{polyfit}\PY{p}{(}\PY{n}{x}\PY{p}{,} \PY{n}{y}\PY{p}{,} \PY{n}{deg}\PY{o}{=}\PY{l+m+mi}{1}\PY{p}{,} \PY{n}{full}\PY{o}{=}\PY{k+kc}{True}\PY{p}{)}
        \PY{n}{b}\PY{o}{.}\PY{n}{append}\PY{p}{(}\PY{n}{b1\PYZus{}hat}\PY{p}{)}
    \PY{k}{return} \PY{n}{b}

\PY{n}{lin\PYZus{}norm} \PY{o}{=} \PY{n}{linear\PYZus{}reg\PYZus{}norm}\PY{p}{(}\PY{l+m+mi}{0}\PY{p}{,} \PY{l+m+mi}{1}\PY{p}{,} \PY{l+m+mi}{1000}\PY{p}{)}
\PY{n}{b\PYZus{}hat\PYZus{}avg} \PY{o}{=} \PY{n+nb}{sum}\PY{p}{(}\PY{n}{lin\PYZus{}norm}\PY{p}{)} \PY{o}{/} \PY{n+nb}{len}\PY{p}{(}\PY{n}{lin\PYZus{}norm}\PY{p}{)}
\PY{n}{plt}\PY{o}{.}\PY{n}{hist}\PY{p}{(}\PY{n}{lin\PYZus{}norm}\PY{p}{)}
\PY{n}{plt}\PY{o}{.}\PY{n}{ylabel}\PY{p}{(}\PY{l+s+s1}{\PYZsq{}}\PY{l+s+s1}{frequency}\PY{l+s+s1}{\PYZsq{}}\PY{p}{)}
\PY{n}{plt}\PY{o}{.}\PY{n}{xlabel}\PY{p}{(}\PY{l+s+s1}{\PYZsq{}}\PY{l+s+s1}{b\PYZus{}hat values}\PY{l+s+s1}{\PYZsq{}}\PY{p}{)}
\PY{n}{plt}\PY{o}{.}\PY{n}{show}\PY{p}{(}\PY{p}{)}
\end{Verbatim}
\end{tcolorbox}

    \begin{center}
    \adjustimage{max size={0.9\linewidth}{0.9\paperheight}}{output_19_0.png}
    \end{center}
    { \hspace*{\fill} \\}
    
    \hypertarget{c}{%
\subsection{C)}\label{c}}

    In this function, I calculated epsilons with the standard cauchy
distribution method rather than the normal distribution. Similar to the
function above, this function calculate 1000 \(\hat\beta_{1}\) and
return a list of the \(\hat\beta_{1}\) values. Taking the values from
the list, I calculated the mean of the 1000 \(\hat\beta_{1}\) generated
using the Cauchy distribution method. After that, I made a histogram
plot for these set of \(\hat\beta_{1}\) values and observed that the
histogram was more centralized around a value while the previous
\(\hat\beta_{1}\) dataset was more spread out.

    The calculated value of the means of the \(\hat\beta_{1}\) coefficients
is 2.196663152813556

    The histogram graph for the 1000 \(\hat\beta_{1}\) with calculating
epsilon using the standard Cauchy distribution method:

    \begin{tcolorbox}[breakable, size=fbox, boxrule=1pt, pad at break*=1mm,colback=cellbackground, colframe=cellborder]
\prompt{In}{incolor}{9}{\boxspacing}
\begin{Verbatim}[commandchars=\\\{\}]
\PY{c+c1}{\PYZsh{} creating the 1000 samples using standard Cauchy distribution }
\PY{k}{def} \PY{n+nf}{linear\PYZus{}reg\PYZus{}cauch}\PY{p}{(}\PY{n}{mu}\PY{p}{,} \PY{n}{sigma}\PY{p}{,} \PY{n}{size}\PY{p}{)}\PY{p}{:}
    \PY{n}{b} \PY{o}{=} \PY{p}{[}\PY{p}{]}
    \PY{n}{np}\PY{o}{.}\PY{n}{random}\PY{o}{.}\PY{n}{seed}\PY{p}{(}\PY{l+m+mi}{2}\PY{p}{)}
    \PY{k}{for} \PY{n}{i} \PY{o+ow}{in} \PY{n+nb}{range}\PY{p}{(}\PY{l+m+mi}{1000}\PY{p}{)}\PY{p}{:}
        \PY{n}{x} \PY{o}{=} \PY{n}{np}\PY{o}{.}\PY{n}{random}\PY{o}{.}\PY{n}{uniform}\PY{p}{(}\PY{n}{mu}\PY{p}{,}\PY{n}{sigma}\PY{p}{,}\PY{n}{size}\PY{p}{)}
        \PY{n}{c} \PY{o}{=} \PY{n}{np}\PY{o}{.}\PY{n}{random}\PY{o}{.}\PY{n}{standard\PYZus{}cauchy}\PY{p}{(}\PY{n}{size}\PY{p}{)}
        \PY{n}{y} \PY{o}{=} \PY{l+m+mi}{5} \PY{o}{+} \PY{l+m+mi}{3}\PY{o}{*}\PY{n}{x} \PY{o}{+} \PY{n}{c}
        \PY{p}{(}\PY{n}{b1\PYZus{}hat}\PY{p}{,} \PY{n}{b\PYZus{}hat}\PY{p}{)}\PY{p}{,} \PY{p}{(}\PY{n}{SSE}\PY{p}{,}\PY{p}{)}\PY{p}{,} \PY{o}{*}\PY{n}{\PYZus{}} \PY{o}{=} \PY{n}{np}\PY{o}{.}\PY{n}{polyfit}\PY{p}{(}\PY{n}{x}\PY{p}{,} \PY{n}{y}\PY{p}{,} \PY{n}{deg}\PY{o}{=}\PY{l+m+mi}{1}\PY{p}{,} \PY{n}{full}\PY{o}{=}\PY{k+kc}{True}\PY{p}{)}
        \PY{n}{b}\PY{o}{.}\PY{n}{append}\PY{p}{(}\PY{n}{b1\PYZus{}hat}\PY{p}{)}
    \PY{k}{return} \PY{n}{b}

\PY{c+c1}{\PYZsh{}Creating the histogram plots for the coefficients created using the standard Cauchy distribution}
\PY{n}{lin\PYZus{}cauch} \PY{o}{=} \PY{n}{linear\PYZus{}reg\PYZus{}cauch}\PY{p}{(}\PY{l+m+mi}{0}\PY{p}{,} \PY{l+m+mi}{1}\PY{p}{,} \PY{l+m+mi}{1000}\PY{p}{)}
\PY{n}{b\PYZus{}hat\PYZus{}avg\PYZus{}cauch} \PY{o}{=} \PY{n+nb}{sum}\PY{p}{(}\PY{n}{lin\PYZus{}cauch}\PY{p}{)} \PY{o}{/} \PY{n+nb}{len}\PY{p}{(}\PY{n}{lin\PYZus{}cauch}\PY{p}{)}
\PY{n+nb}{print}\PY{p}{(}\PY{n}{b\PYZus{}hat\PYZus{}avg\PYZus{}cauch}\PY{p}{)}
\PY{n}{plt}\PY{o}{.}\PY{n}{hist}\PY{p}{(}\PY{n}{lin\PYZus{}cauch}\PY{p}{)}
\PY{n}{plt}\PY{o}{.}\PY{n}{ylabel}\PY{p}{(}\PY{l+s+s1}{\PYZsq{}}\PY{l+s+s1}{frequency}\PY{l+s+s1}{\PYZsq{}}\PY{p}{)}
\PY{n}{plt}\PY{o}{.}\PY{n}{xlabel}\PY{p}{(}\PY{l+s+s1}{\PYZsq{}}\PY{l+s+s1}{b\PYZus{}hat values}\PY{l+s+s1}{\PYZsq{}}\PY{p}{)}
\PY{n}{plt}\PY{o}{.}\PY{n}{show}\PY{p}{(}\PY{p}{)}
\end{Verbatim}
\end{tcolorbox}

    \begin{Verbatim}[commandchars=\\\{\}]
2.196663152813556
    \end{Verbatim}

    \begin{center}
    \adjustimage{max size={0.9\linewidth}{0.9\paperheight}}{output_24_1.png}
    \end{center}
    { \hspace*{\fill} \\}
    
    \hypertarget{problem-3}{%
\section{Problem 3}\label{problem-3}}

    \hypertarget{a}{%
\subsection{a)}\label{a}}

    \$x\_\{1\} = \$ 40 hours of studying    \$x\_\{2\} = \$ 3.5 GPA    Y =
Recieving an A    \(\hat\beta_{0}\) = -6    \$\hat\beta\emph{\{1\} = \$
0.05    \$\hat\beta}\{2\} = \$ 1 \(\hat P\)(Y = A \textbar{} \(X_{1}\)=
40 hours of studying, \(X_{2}\) = 3.5 GPA) =
\(\Large\frac{e^{\hat\beta_{0} + \hat\beta_{1}x_{1} + \hat\beta_{2}x_{2}}}{1 + e^{\hat\beta_{0} + \hat\beta_{1}x_{1} + \hat\beta_{2}x_{2}}}\)
=
\(\Large\frac{e^{(-6) + (0.05)(40) + (1)(3.5)}}{1 + e^{(-6) + (0.05)(40) + (1)(3.5)}}\)
= \(0.3775\)

The probability of the student getting an A while studying for 40 hours
and having a 3.5 GPA is 37.75\%.

    \hypertarget{b}{%
\subsection{b)}\label{b}}

    \$x\_\{1\} = \$ ?    \$x\_\{2\} = \$ 3.5 GPA    Y = Recieving an A   
\(\hat\beta_{0}\) = -6    \$\hat\beta\emph{\{1\} = \$ 0.05   
\$\hat\beta}\{2\} = \$ 1    \(\hat P\)(Y = Getting an A\textbar{} X1,
X2) = 0.5 \(\hat P\)(Y = A \textbar{} \(X_{1}\)= ?, \(X_{2}\) = 3.5 GPA)
= 0.5
\(\normalsize = \frac{e^{(-6) + (0.05)X_{1} + (1)(3.5)}}{1 + e^{(-6) + (0.05)X_{1} + (1)(3.5)}} = 0.5\)

\(\normalsize e^{(-6) + (0.05)X_{1} + (1)(3.5)} = 0.5 + 0.5e^{(-6) + (0.05)X_{1} + (1)(3.5)}\)
\(\normalsize e^{(-6) + (0.05)X_{1} + (1)(3.5)} = 1\)
\(\small ln(e^{(-6) + (0.05)X_{1} + (1)(3.5)}) = ln(1)\)
\(\small (-6) + (0.05)X_{1} + (1)(3.5) = 0\)
\(\small (0.05)X_{1} = 2.5\) \(\normalsize X_{1} = \frac{2.5}{0.05}\)
\(\normalsize X_{1} =\) 50 hours of studying The students need to study
at least 50 hours in order to get a 50\% chance of getting an A in the
course

    \hypertarget{problem-4}{%
\section{Problem 4}\label{problem-4}}

    P(X = 1\textbar{} Y = 0) = 1/3    P(X = 1\textbar{} Y = 1) = 0

P(X = 2\textbar{} Y = 0) = 2/3    P(X = 2\textbar{} Y = 1) = 1/3

P(X = 3\textbar{} Y = 0) = 0    P(X = 3\textbar{} Y = 1) = 2/3

P(Y = 0\textbar{} X = 1) =
\(\frac{P(X = 1|Y = 0)P(Y = 0)}{P(X = 1|Y = 0)P(Y = 0) + P(X = 1|Y = 1)P(Y = 1)}\)
= \(\frac{(1/3)(1/2)}{(1/3)(1/2) + (0)(1/2)}\) = 1

P(Y = 0\textbar{} X = 2) =
\(\frac{P(X = 2|Y = 0)P(Y = 0)}{P(X = 2|Y = 0)P(Y = 0) + P(X = 2|Y = 1)P(Y = 1)}\)
= \(\frac{(2/3)(1/2)}{(2/3)(1/2) + (1/3)(1/2)}\) = \(\frac{2}{3}\)

P(Y = 0\textbar{} X = 3) =
\(\frac{P(X = 3|Y = 0)P(Y = 0)}{P(X = 3|Y = 0)P(Y = 0) + P(X = 3|Y = 1)P(Y = 1)}\)
= \(\frac{(0)(1/2)}{(0)(1/2) + (2/3)(1/2)}\) = 0

P(Y = 1\textbar{} X = 1) =
\(\frac{P(X = 1|Y = 1)P(Y = 1)}{P(X = 1|Y = 0)P(Y = 1) + P(X = 1|Y = 0)P(Y = 0)}\)
= \(\frac{(0)(1/2)}{(0)(1/2) + (1/3)(1/2)}\) = 0

P(Y = 1\textbar{} X = 2) =
\(\frac{P(X = 2|Y = 1)P(Y = 1)}{P(X = 2|Y = 0)P(Y = 1) + P(X = 2|Y = 0)P(Y = 0)}\)
= \(\frac{(1/3)(1/2)}{(1/3)(1/2) + (2/3)(1/2)}\) = \(\frac{1}{3}\)

P(Y = 1\textbar{} X = 3) =
\(\frac{P(X = 3|Y = 1)P(Y = 1)}{P(X = 3|Y = 0)P(Y = 1) + P(X = 3|Y = 0)P(Y = 0)}\)
= \(\frac{(2/3)(1/2)}{(2/3)(1/2) + (0)(1/2)}\) = 1

\(f^{*}(X=1)\) = 1 when Y = 0 \(f^{*}(X=2)\) = \(\frac{2}{3}\) when Y =
0 \(f^{*}(X=3)\) = 1 when Y = 1

    Please sign below (Truc Le) after checking (\(\checkmark\)) the
following. If you can not honestly check each of these responses, please
email me at kbala@ucdavis.edu to explain your situation. * We pledge
that we are honest students with academic integrity and we have not
cheated on this homework.

\begin{itemize}
\item
  These answers are our own work.
\item
  We did not give any other students assistance on this homework.
\item
  We understand that to submit work that is not our own and pretend that
  it is our is a violation of the UC Davis code of conduct and will be
  reported to Student Judicial Affairs.
\item
  We understand that suspected misconduct on this homework will be
  reported to the Office of Student Support and Judicial Affairs and, if
  established, will result in disciplinary sanctions up through
  Dismissal from the University and a grade penalty up to a grade of
  ``F'' for the course.
\end{itemize}

Team Member 1: Truc Le


    % Add a bibliography block to the postdoc
    
    
    
\end{document}
